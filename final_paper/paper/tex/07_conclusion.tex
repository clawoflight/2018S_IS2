\section{Conclusion} % ~ 3/4 page
\label{sec:conclusion}

% work out requirements and summarize results; countermeasures; outlook
IoT devices in general are largely insecure and should be trusted at most as much as any other device in a network.
Smart lights in particular are interesting because they are networked LEDs and therefore could be used to build a covert communication channel.
Additionally, visible light is often the only part of the EM~spectrum that is allowed to leave an air-gapped system.
Therefore, the attack that we replicated in this paper could potentially be used to exfiltrate data from air-gapped network.

Ronen~and~Shamir~\cite{Ronen:2016:EFAIDCSL} showed that specially-crafted PWM profiles could be used to build a covert channel and mentioned that the PWM from dimming could be used once the method they used was no longer available.
Our findings confirm that argument: Using the PWM, we were able to distinguish brightness levels indistinguishable to the human eye.
Therefore, Smart LEDs lend themselves to building covert channels.

More work is required to make this attack practical, but it is definitely feasible even with limited resources.
Given better signal processing, calibration, and a good channel encoding, the full transmission sequence could be automated in a very reasonable time frame. 
% TODO this may be better somewhere else?
To make the attack truly practical, better range would be required, for example by putting the light sensor in the focal point of a telescope focused on the light bulb~\cite{Ronen:2016:EFAIDCSL} and maybe using a more sensitive sensor.
That telescope would be the most expensive required piece of hardware by several orders of magnitude. \\

The biggest takeaway from this is that smart lights are not to be trusted in secure environments. 
%Use dumb lights instead.
Better use traditional lightning solution within sensitive environments.
If for some reason it is necessary to use networked lights, standard security practices (for networked and IoT devices) should be followed:
Change passwords regularly, disable unused services, isolate IoT devices onto their own network(s) or use specially tailored engines to curtail the IoT devices' controlling rights, and raise awareness that these devices are networked general-purpose computing systems.
Additionally, seriously consider to not allow visible light to exit the air-gapped room or building by designing it without windows.

% Countermeasurese
% general:
% change authentication mechanisms (dont use default pws, ...)
% update security measure regularly
% disable unused ports (spec for botnet prevention)
% isolate iot devices on own rotected network
% end users need to be aware of capabilities and apps installed and interactiong with iot devices
% ...
% spec to light:
% no windows
% ...
