\section{Related Work} % ~ 2 pages
\label{sec:related_work}

% make clear what is relevant for our topic and what is not, put some more technical details to parts which are relevant for our project --> only mention relevant papers (VLC and Ronen/Shamir)

%Ronen/Shamir
Ronen~and~Shamir~\cite{Ronen:2016:EFAIDCSL} have found out, that sensitive data like i.e.\ passwords and keys can be extracted even from air-gapped networks using IoT light bulbs within the same network and the principal of visible~light~communication~(VLC). The creation of a covert communication channel was tested on two different IoT lightning solutions. One was the widely known Philips Hue White lightning system, former called Philips Lux. 
%The other was LimitlessLED which is one of the cheap smart light solutions on the market.
The idea is to switch between two very close brightness levels which further allows to interpret the corresponding pulse~width~modulation~(PWM) signal as logical zero and logical one, respectively. Philips comes with 255 different brightness levels and uses a PWM frequency around 20~KHz. 
%LimitlessLED in contrast has only 24 different levels of brightness using a PWM frequency of around 3 KHz. 
%Regarding Philips light bulbs t
This leads to measure differences of only 200~nanoseconds and thus high end measuring equipment is needed including sensible light sensors in order to actually see the difference in luminosity.
When changing brightness levels a measurable phase shift is created in the light sensor's frequency output which allows to further decode the sent data.
Assuming that light in offices is turned on for at least 12~hours a day, around 10~KB each day could be leaked which is actually enough for i.e.\ passwords and keys.

%covert router led channel
The general idea of using light~emitting~diodes~(LEDs) for data exfitration from air-gapped networks was also conducted by Guri~et.~al~\cite{Guri:2017:xCDEANvRL}. Their idea was to control the router's activity and status LEDs via a malicious script running on the router which accesses the routers General~Purpose~Input~Output~(GPIO) controls in order to change the LEDs blinking pattern according to the data which an potential attacker wants to extract.
To be able to receive the data the LEDs need to be in the line of sight of the attacker. Therefor Guri~et.~al tested two different approaches. For one they used a video camera to determine on and off periods. When using a camera the maximal bit rate depends on the type of camera, i.e.\ the best possible solution is using a extreme camera as for example a GoPro. With this a transmission rate of 800-960~bit~per~second~(bit/s) can be achieved when using eight LEDs. A much higher bit rate could be obtained using an optical sensor, as Ronen~and~Shamir~\cite{Ronen:2016:EFAIDCSL} did. In that case Guri~et.~al were able to transmit 4000 bit/s, also using eight LEDs.

%vlc in general
VLC in general is of high interest nowadays and may also be in future since it solves some of the limitations we meet with traditional wireless transmission means. For one it is less expensive since no radio~frequency~(RF) units are needed, further the light signal is free of any health concerns, it is also conductible in RF sensitive areas, i.e.\ planes, and the bandwidth is not limited, and so on~\cite{Elgala:2007:OVLWCBoWL}. Further, with the rise of IoT and thus also smart light solutions, LED bulbs will again become more and more prevalent.
VLC has by now been studied for decades. By that time various different modulation schemes exist in order to encode data. Indeed, the most standing to reason sort of modulation is the On-Off-Keying~(OOK) modulation scheme, but on the other hand OOK brings limitations on the conductible data rates. Several researches have thus been made on Orthogonal Frequency-Division Multiplexing (OFDM) together with higher level modulation schemes~\cite{Elgala:2007:OVLWCBoWL,Yu:2014:BCDRCVLOS}. With IoT light bulbs the modulation issues can be disregarded, since the PWM signal cannot be modulated directly. But, as it was already described in the beginning of this section, we can still reach enough bandwidth for extracting sensitive data. 