\section{Communication with Light} % ~ 1 to 1.5 page(s)
\label{sec:light_communication}

Before we can look at how a functionality extension attack on a smart light system can actually be leveraged, we need to understand how communication over light works.

%description of required knowledge in order to be able to understand and conduct attack (genaral communication with light)

In VLC~\cite{Komine:2004:FAfVLCSuLL, Yu:2014:BCDRCVLOS} the visible part of the electromagnetic~(EM)~spectrum, namely the visible light, is used for communication purposes. VLC is a subset of optical wireless communication technologies, like infrared.
In order to conduct VLC, simple white LED bulbs are needed. The actual transmission of data is based on the fact that LEDs can be switched on and off at such a high rate, that those intensity changes cannot be seen by the human eye.

As known from the video game section, human eyes are capable of seeing flickers above 30~Hz. The difference between 30~frames~per~second~(fps) and 60~fps can easily be noticed. Flicker rates which go beyond 60~Hz can though not be detected by the human eye. So in the context of light communication, especially covert light communication as needed in our case, it is important to flicker at over 60~Hz. Those quick flickers allow constant illumination besides the ability of data transmission.

The flickers are done by rapidly switching between on and off states and adjusting the duty cycle. The transmission of data is further allowed by interpreting the on period as logical one and the off period as logical zero. In order to actually get the encoded data out of the light signal the duration and frequency of the light flickers need to be measured. This is done using a light sensor since these are capable of accurately distinguishing between the on and off states and measuring the duty cycle. Further light sensors are robust to other light sources or any other noise.

In the case of smart lights, the PWM signal cannot be changed directly, since the light intensity is changed by sending commands over the manufacturers Application~Programming~Interface~(API). Those commands internally modulate the pulse width of the light signal. Thus, by sending two close brightness commands, we can achieve the same effect as with traditional VLC and covertly transmit data. Different to traditional VLC, a logical one is represented by the higher brightness level whilst a logical zero is represented by the lower brightness level. Fortunately, since internally the luminosity of the LEDs is again changed by adjusting the PWM signal, light sensors are again capable of measuring the differences. 
