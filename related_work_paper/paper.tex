\documentclass[11pt,a4paper]{article}

\begin{document}
	\title{IoT Light Bulb Covert Channel and Other Functionality Extension Attacks: Related Work}
	\author{Julia Wanker, Bennett Piater}
	\date{\today}
	\maketitle

	\section{Security of IoT Devices}%
	\label{sec:security_of_iot_devices}

	\subsection{General Security}% NOTE: Proofread and shortened
	\label{sub:general_security}

	% \cite{Grau:2015:Ctyf} % Fridge and more % magazine article. nothing new here, good to make non-scientists aware of the issues though.
	% \cite{Bhartiya::YSFMK} % Fridge and more % online article. nothing new here, except the idea that a fridge or freezer could be used for murder.

  It is widely known that IoT devices have poor security in general.
	The most recent state-of-the art security survey was performed by
	Zhang~et~al.~\cite{Zhang:2017:UISTDCBWWNaWWG} % General
	They provide a detailed analysis of vulnerabilities and defence mechanisms.
	In particular, they note that much academic literature is overly conservative because most security analyses are published in whitepapers and on blogs, causing them to be ignored in scientific surveys.
	They suspect either a lack of expertise, or outright neglect of security design on the part of the vendors.

	Additionally,
	Restuccia~et~al.~\cite{Restuccia:2018:SITNPaRC} % General
	recently provided a very good analysis and taxonomy of the systematic problems and future challenges of IoT security. The paper strongly advocates for security by design of connected devices from their cradle to their grave.

	\subsection{Smart Light Security}% TODO - check and ev shorten
	\label{sub:smart_light_security}
	The security of smart light systems is of decisive importance since smart light solutions are also used in public buildings and offices. Hence there are several former studies analyzing the security of IoT light bulbs~\cite{Dhanjani:2013:HLSEPHPWLS, Morgner:2016:AYBBUICSSCLS, Ronen:2018:IGNCZCR}.

	% Light bulbs
	Dhanjani~\cite{Dhanjani:2013:HLSEPHPWLS} found several ways to initiate \textit{Denail-of-Service}~(DoS) attacks. He was able to cause sustained but also perpetual blackouts which can be of high risk i. e. if hospitals are involved. Those blackouts could be conducted due to insufficient security measures when connecting a smart light bulb with a controlling device. Dhanjani also mentioned the possibility of encryption flaws in the implementation of the \textit{ZigBee Light Link}~(ZLL) which is used for the communication between the bridge and the light bulbs. But due to Dhanjani's findings the ZLL communication could only be intercepted if the attacker is physically close and thus such an attack would not be of high risk.

	% Light bulbs & ZLL
	Morgner~et~al.~\cite{Morgner:2016:AYBBUICSSCLS} further investigated the fact of smart light systems being insecure due to ZLL and could show that this sort of attack is also of high importance. They were able to control ZLL-certified light bulbs from a distance over 15 to 36 meters. Their research prove and particularized Dhanjani's~\cite{Dhanjani:2013:HLSEPHPWLS} findings that exploitable vulnerabilities exist by the design of the ZLL standard. The ZLL standard provides the so called \textit{touchlink~commissioning} which employs a global ZLL master key to secure the setup process. This master key was leaked in 2015~\cite{Morgner:2016:AYBBUICSSCLS} and ever since the touchlink procedure is considered to be insecure. Due to the flaws in the touchlink specification Morgner~et~al. were able to introduce a new network key which was then accepted by all connected light bulbs which further allowed the authors to send malicious commands.

	% ZLL
	Ronen~et~al.~\cite{Ronen:2018:IGNCZCR} also used flaws in ZLL to attack smart light solutions. Their attack was of even higher concern since they were able to exchange the light bulbs firmware with a malicious malware and because of vulnerabilities in the ZigBee communication they were able to further spread the malware over all approximate light bulbs. Thus an attacker would be able to launch a \textit{war-flight} and infect all smart lamps of a whole city.

	\section{Functionality-Ignoring Attacks}% NOTE: proofread and shortened. Not sure we should keep it though!
	\label{sec:functionality_ignoring_attacks}
	% Focus on botnets for spam, ddos
	A big portion of the research on IoT security was conducted about attacks ignoring the intended functionality of IoT devices. In particular, the appearance of the Mirai botnet led to multiple papers about botnets comprised of IoT devices.

	Angrishi~\cite{Angrishi:2017:TitiiiviIb} % botnets
	makes the very important point that IoT devices should not be seen as specialized devices with added intelligence, but rather as (general) computing devices that are performing specialized tasks.
	Attackers are certainly aware of this, and most attacks on IoT devices involve botnets for DDoS or spamming. DDoS-capable malware was surveyed and classified by Donno~et~al.~\cite{Donno:2017:ADIM}.

	The most comprehensive analysis of the Mirai botnet, responsible for the record-breaking 1.3Tb/s DDoS on DynDNS, was published by
	Antonakakis~et~al.~\cite{Antonakakis:2017:UMB}. % Mirai
	In particular, they found a list of default passwords found in the source code of the malware, which clearly show it targeting cheap IoT devices, many of them IP cameras.
	They clearly show that Mirai succeeded primarily because of incredibly low-hanging fruit: (tiny) dictionary attacks on devices accessible from the open internet were enough.

	% \cite{Kolias:2017:DIMaOB} % botnets % removed because strictly inferior to Antonakakis:2017:UMB

	\section{Functionality-Extending Attacks}% TODO - check and ev shorten
	\label{sec:functionality_extending_attacks}
	The most interesting sort of attack is the so called \textit{Functionality-Extending~Attack} where an attacker uses the IoT lightbulb for other purposes than illumination. In particular an attacker can use light~emitting~diodes~(LEDs) for an optical wireless communication system, which was elaborated several years ago~\cite{Komine:2004:FAfVLCSuLL, Elgala:2007:OVLWCBoWL}. Since smart light solutions use LEDs, Ronen~and~Shamir~\cite{Ronen:2016:EFAIDCSL} were able to create a covert communication channel in association with their security analysis of IoT devices. As the setup process of an IoT light bulb is vulnerable~\cite{Dhanjani:2013:HLSEPHPWLS, Morgner:2016:AYBBUICSSCLS, Ronen:2018:IGNCZCR}, Ronen and Shamir were able to abuse the \textit{application~programming~interface}~(API) of the IoT light bulb in order to let the LEDs switch between two light intensities at a very high rate, such that it cannot be noticed by the human eye but can be detected by a light sensor. The light sensor measures the exact duration and frequency of those flickers and converts it to a digital frequency in order to leak sensitive data. Ronen and Shamir could show that this kind of attack is also possible on \textit{protected} networks, i. e. air-gapped networks. Besides leaking data through a covert channels they have shown that the light flickering can also be misused for creating epileptic seizures.

	\newpage

	\bibliographystyle{plain}
	\bibliography{../papers/literature.bib}
\end{document}
% vim: sw=2 ts=2 spell
