%% Begin slides template file
\documentclass[11pt,t,usepdftitle=false,aspectratio=169]{beamer}
%% ------------------------------------------------------------------
%% - aspectratio=43: Set paper aspect ratio to 4:3.
%% - aspectratio=169: Set paper aspect ratio to 16:9.
%% ------------------------------------------------------------------
\usepackage{graphicx}
\usepackage{hyperref}
\usetheme[nototalframenumber,logo,license]{uibk}

%% NOTES
\usepackage{pgfpages}
\setbeameroption{show notes on second screen=right} % Both


%% ------------------------------------------------------------------
%% - foot: Add a footer line for conference name and date.
%% - logo: Add the university logo in the footer (only if 'foot' set).
%% - bigfoot/sasquatch: Larger font size in footer.
%% - nototalslidenumber: Hide the total number of slides (only if 'foot' set)
%% - license: Add CC-BY license symbol to title slide (e.g., for conference uploads)
%%   (TODO: At the moment no other licenses are supported.)
%% - licenseall: Add CC-BY license symbol to all subsequent slides slides
%% - url: use \url{} rather than \href{} on the title page
%% ------------------------------------------------------------------

%% ------------------------------------------------------------------
%% The official corporate colors of the university are predefined and
%% can be used for e.g., highlighting something. Simply use
%% \color{uibkorange} or \begin{color}{uibkorange} ... \end{color}
%% Defined colors are:
%% - uibkblue, uibkbluel, uibkorange, uibkorangel, uibkgray, uibkgraym, uibkgrayl
%% The frametitle color can be easily adjusted e.g., to black with
%% \setbeamercolor{titlelike}{fg=black}
%% ------------------------------------------------------------------

%\setbeamercolor{verbcolor}{fg=uibkorange}
%% ------------------------------------------------------------------
%% Setting a highlight color for verbatim output such as from
%% the commands \pkg, \email, \file, \dataset
%% ------------------------------------------------------------------


%% information for the title page ('short title' is the pdf-title that is shown in viewer's titlebar)
\title[IoT Light Bulb Attack]{IoT Light Bulb Covert Channel}
\subtitle{Extended Functionality Attack on Smart Lights}
\URL{}

\author[Julia Wanker \& Bennett Piater]{Julia Wanker, Bennett Piater}
%('short author' is the pdf-metadata Author)
%% If multiple authors are required and the font size is too large you
%% can overrule the font size of author and url by calling:
%\setbeamerfont{author}{size*={10pt}{10pt},series=\mdseries}
%\setbeamerfont{url}{size*={10pt}{10pt},series=\mdseries}
%\URL{}
%\subtitle{}

\footertext{}
\date{2018-06-14}

\headerimage{3}
%% ------------------------------------------------------------------
%% The theme offers four different header images based on the
%% corporate design of the university of innsbruck. Currently
%% 1, 2, 3 and 4 is allowed as input to \headerimage{...}. Default
%% or fallback is '1'.
%% ------------------------------------------------------------------

\begin{document}

%% ALTERNATIVE TITLEPAGE
%% The next block is how you add a titlepage with the 'nosectiontitlepage' option, which switches off
%% the default behavior of creating a titlepage every time a \section{} is defined.
%% Then you can use \section{} as it's originally intended, including a table of contents.
% \usebackgroundtemplate{\includegraphics[width=\paperwidth,height=\paperheight]{titlebackground.pdf}}
% \begin{frame}[plain]
%     \titlepage
% \end{frame}
% \addtocounter{framenumber}{-1}
% \usebackgroundtemplate{}}

%% Table of Contents, if wanted:
%% this requires the 'nosectiontitlepage' option and setting \section{}'s as you want them to appear here.
%% Subsections and subordinates are suppressed in the .sty at the moment, search
%% for \setbeamertemplate{subsection} and replace the empty {} with whatever you want.
%% Although it's probably too much for a presentation, maybe for a lecture.
% \begin{frame}
%     \vspace*{1cm plus 1fil}
%     \tableofcontents
%     \vspace*{0cm plus 1fil}
% \end{frame}

%%%%%%%%%%%%%%%%%%%%%%%%%%%%%%%%%%%%%%%%%%%%%%%%%%%%%%%%%%%%%%%%%%%%%%%%%
% Intro:
% - names
% - topic
% - structure: First Taxonomy (main paper), then overview of main paper and possible related work
%%%%%%%%%%%%%%%%%%%%%%%%%%%%%%%%%%%%%%%%%%%%%%%%%%%%%%%%%%%%%%%%%%%%%%%%%

\section{Topic Relevance} % 5 min
\label{sec:relevance}
\title{Topic Relevance}
\subtitle{New Attack Vectors on IoT Devices}

\subsection{IoT Security in General} % 1.5 min
\label{sub:general_security}
\begin{frame}{IoT Security in General}

\end{frame}

\subsection{Smart Light Security} % 1 min
\label{sub:smart_light_security}
\begin{frame}{Smart Light Security}

\end{frame}

\subsubsection{Ignoring Functionality} % 1 min
\label{subsub:ignoring}
\begin{frame}{Ignoring Functionality}
	
\end{frame}

\subsubsection{Extending Functionality} % 1.5 min
\label{subsub:extending}
\begin{frame}{Extending Functionality}
	
\end{frame}


\section{Theoretical Background} % 7 min
\label{sec:theory}
\title{Theoretical Background}
\subtitle{Extended Functionality Attack on Smart Lights}

\subsection{Covert Communication Channel} % 1 min
\label{sub:covert_channel}
\begin{frame}{Covert Communication Channel}

\end{frame}

\subsection{Communication With Lights} % 1 min
\label{sub:light_communication}
\begin{frame}{Communication With Lights}

\end{frame}

\subsection{Smart Light Systems} % 1 min
\label{sub:smart_lights}
\begin{frame}{Smart Light Systems}
  
\end{frame}

\subsubsection{Encoding} % 2 min
\label{subsub:encoding}
\begin{frame}{Encoding: ...}
     
\end{frame}

\subsubsection{Decoding} % 2 min
\label{subsub:decoding}
\begin{frame}{Decoding: ...}
 
\end{frame}

\section{Demonstration} % 5 min
\label{sec:demonstration}
\title{Demonstration}
\subtitle{Covert Communication Channel on Philips Hue White Smart Light}

\section{Conclusion} % 3 min
\label{sec:conclusion}
\title{Conclusion}
\subtitle{Summary and Outlook}
\begin{frame}{Conclusion} % 1.5 min

\end{frame}

\begin{frame}{Outlook} % 1.5 min

\end{frame}



%% to show a last slide similar to the title slide: information for the last page
\title{Questions?}
\subtitle{}
\section{Questions}


%% appendix of 'extra' slides
\appendix

\begin{frame}[allowframebreaks]{Bibliography}
	\bibliographystyle{alpha}
	\bibliography{../papers/literature.bib}
\end{frame}

\end{document}

% vim: spell ts=2 sw=2
